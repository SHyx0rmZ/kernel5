\documentclass[a4paper,openany]{scrbook}
\usepackage{xunicode}
\usepackage{xltxtra}
\usepackage{xeCJK}
\setCJKmainfont{Sazanami Gothic}

\usepackage{listings}
\usepackage{fancyhdr,afterpage}
\usepackage{hyperref}
\usepackage{ulem}

\subject{Everything you need to know and more about:}
\title{The Nuke Operating System\\kernel5 - 理コ込}
\author{Patrick Pokatilo}
\date{2010.10.21}

\begin{document}
\normalem

\pagestyle{fancy}
\setlength{\oddsidemargin}{0mm}
\setlength{\evensidemargin}{0mm}
\setlength{\textwidth}{159.2mm}
\setlength{\parskip}{\medskipamount}
\setlength{\parindent}{0mm}

\begin{titlepage}
\maketitle
\end{titlepage}

\tableofcontents

\part{Introduction}
\chapter{Prologue}
Despite being only 18 years old Nuke has already been a big part of
my life. I first had plans for my own operating system when I was
15 years old, and I have written several ``prototypes'' sinces then.
I failed many times and tossed the oriject away more than once. But
especially in the last year, I think I have learned many important
things and gained quite some experience.

I have always wanted Nuke to become an Exokernel, despite not knowing
how to do this exactly. While the most critical design goal was speed
in the earlier years, I have decided to concentrate on stability and
safety for now, Because Nuke was designed for speed its first versions
were written in Assembly, but I soon listened to
the kind people\footnote{\tt{irc://irc.euirc.net:6667/\#lost}}
who always told me to ``write in C''. Well, not exactly. I did try
C for a \textsl{very} tiny version of Nuke but soon tossed it in favor
of C++. While I certainly don't claim to write good code (nobody does,
right?) it proved good enough to get through quite a lot of the problems
that I didn't manage to solve in earlier versions.

This continued well until I wanted to refine my implementation of
a memory manager. I couldn't quite grasp the correct solution and
soon found my C++ code to be bloated. Therefore I decided to rewrite
the kernel from scratch (yet again). This time I would write it in C and, for
the first time, provide not only an amd64 build, but also an i386 build.
Well, that's how kernel5 came to life. The next chapters shall define
my new design goals and details of implementations, so that I won't
forget how some parts work together. I'll also comment the code well,
so that you can follow my thought processes and learn something from
this project.

\part{Implementations}
\chapter{Memory Management}
Memory is managed through Paging, therefore it is divided into physical
and virtual memory management.
\section{Physical Memory Management}
The physical memory management stores blocks of arbitrary sizes
in two doubly linked lists. There is one list for allocated blocks
and one list for free blocks. I use lists so I don't have to worry about
where to save memory management data, as it can simply be stored into the
beginning of each memory blocks. When \texttt{memory\_init()} is called, the
two lists are initialized according to the multiboot memory map. Note that
used memory blocks may be unfreeable. \emph{Trying to free an unfreeable
memory block will terminate the calling process.}

\begin{lstlisting}[basicstyle={\ttfamily},breaklines=true,language=C,numbers=right]
typedef struct memory_area
{
    uintptr_t address;
    size_t size;
} memory_area_t;

memory_area_t *memory_alloc(size_t size, uintptr_t limit, uintptr_t align);
void memory_free(memory_area_t *area);
\end{lstlisting}
\section{Virtual Memory Management}
Is not yet implemented and probably never will be. It most likely will be
done by the libOS.
\part{Syscall Interface}
\chapter{Uses}
As Nuke is (supposed to be) an Exokernel it relies on syscalls to
provide applications and libraries with the functionality they need
to implement a minimal runtime environment. The syscall interface
enables programs from userspace to request and manage resources.
\chapter{List of Syscalls}
\section{\texttt{memory\_area\_t *memory\_alloc(size\_t size, uintptr\_t limit, uintptr\_t align)}}
Request a memory area of minimum size \emph{size}. The area's end (address
+ size) must not lie above \emph{limit}. Use a \emph{limit} of 0 to not limit
the result. The memory area is aligned on \emph{align} bytes. \textbf{Always
use a power of 2 for \emph{align}}. Set \emph{align} to 0 to ignore the
alignment.
\section{\texttt{void memory\_free(memory\_area\_t *area)}}
Free a memory area. If area is not a valid memory area (wasn't supplied by
\texttt{memory\_alloc()}), kill the process.
\part{Additions}
\chapter{Known Bugs}
\begin{itemize}
\item \texttt{memory\_init()} may crash due to a \#PF, when a memory area starts at a high enough address.
\item \texttt{memory\_alloc()} will return complete garbage if called with a
value for \emph{align} that is not a power of 2.
\end{itemize}
\end{document}
